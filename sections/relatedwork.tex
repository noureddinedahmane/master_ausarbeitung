%% LaTeX2e class for student theses
%% sections/conclusion.tex
%% 
%% Karlsruhe Institute of Technology
%% Institute for Program Structures and Data Organization
%% Chair for Software Design and Quality (SDQ)
%%
%% Dr.-Ing. Erik Burger
%% burger@kit.edu
%%
%% Version 1.3.2, 2017-08-01

\chapter{Related Work}
\label{ch:Related Work}

None of the existing approaches that deal with the automatic extraction of performance
model considers the parametric dependencies of performance models, the adaptive
monitoring and the incremental execution of the whole process at the same time. Moreover,
existing approaches do not support the adaptive monitoring for performance model continuous
integration.

Walter et al. \cite{walter2017expandable} provided an expandable approach for automated extraction of
architectural performance model. He presents a framework for the extraction of
the architectural performance models based on monitoring files generalizing over
the target architectural model language. This framework can be used to create
an extraction tool for a specific modeling formalism. He implemented his approach
in so called Performance Model Extractor (PMX) tool and provided it as a web service.

Krogmann et al. \cite{krogmann2012reconstruction, krogmann2010using} used genetic programming to estimate parametric dependencies between inputs and the number of execution for each byte code instruction.
Depending on input data he could find control flow dependencies between required
and provided services. However, for receiving monitoring information, he instruments the whole system in each iteration, which causes a monitoring overhead. 

Spinner et al. \cite{spinner2016reference} predicted the system performance at run time under varying work
loads and system configuration. He proposed an approach for obtaining architectural-level model performance in virtualized environment.

Langhammer \cite{langhammer2017automated} introduced two approaches to extract the static and the dynamic
behavior of the source code. Furthermore, he proposed an approach that keeps the source code and the 
architecture consistent during development. Furthermore, he could generate the performance model incrementally. However, he did not use an adaptive monitoring approach, which means he instruments the whole system in each iteration. Therefore, the monitoring overhead could not be reduced.

Brosig et. al \cite{brosig2009automated} introduce an approach that the Palladio Component Models of of Java EE applications based on the monitoring data collected during operation. However, he did take into account the adaptive monitoring, which means he monitor the whole source code in each iteration. Therefore, monitoring huge Java EE applications will lead to monitoring overhead.    

Jägers has also introduced in his master thesis \cite{jägers2018} an approach that estimates incrementally the performance model parameters considering parametric dependencies. Moreover, he used fine-grained monitoring information for the estimation. However, he instrumented the source code manually. 
