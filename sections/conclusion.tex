%% LaTeX2e class for student theses
%% sections/conclusion.tex
%% 
%% Karlsruhe Institute of Technology
%% Institute for Program Structures and Data Organization
%% Chair for Software Design and Quality (SDQ)
%%
%% Dr.-Ing. Erik Burger
%% burger@kit.edu
%%
%% Version 1.3.2, 2017-08-01

\chapter{Conclusions and Future Work}
\label{ch:Conclusions and Future Work}

This chapter presents a conclusion of our approach and introduces suggestions for future work. 

\subsection{Summary}
\label{sec:summary}
We presented an approach that monitors the source code adaptively and provides performance models with the needed monitoring information for estimating the performance models parameters. Our approach is part of the continuous integration of performance model, presented by Mazkatli and Koziolek. we can provide fine-grained monitoring information that are needed for estimating Palladio Component Model like the number of loop execution, the probability of selecting a branch and resource demands. Moreover, we can provide the external calls arguments, which can be used to estimate performance model parameters considering parametric dependencies.

We extended the Vitruvius VSUM with an Instrumentation Meta-model that can be used to describe and manage probes for source code instrumentation. The generation of these probes is based on source code changes. Moreover, we created a transformation that keep the source code and the Instrumentation model consistent.

We provided an adaptive and automatic instrumentation approach that instruments the source code iteratively and automatically. We based our instrumentation on the Kiekeker Monitoring Framework. We extended Kieker with new monitoring records that contains the monitoring information needed for performance model parameters estimation.  

We provided an adaptive monitoring approach, which is based on the activeness of the probes. That means, the probes can be activated and deactivated during monitoring phase, so that only the activated probes will be monitored.

We evaluate our approach in three levels. In the first level, we showed that we could instrument the source code incrementally. In the second level, we showed that the monitoring information that we generated can be used for performance model parameters estimation. In the third level, we showed that we could reduce the monitoring overhead to 50\% in our case study. 
   
\subsection{Future Work}
\label{sec:future work}
We identify two disadvantages of our approach. The first one is related to the filtering of the probes that have no effect on the performance model. These probes are for example internal actions that contain only one variable declaration, which has less response time then logic used to instrument them. In our approach. we filter these probes after the monitoring phase, which helps to reduce the monitoring records number and thus reduce the time used for the performance model parameters estimation. Therefore, we suggest to create in a future work, an approach that identifies these probes based on a static analysis of the source code. Such an approach will also reduce the monitoring overhead. 

The second disadvantage is related to level of adaptive monitoring in our approach. As we mentioned above, our incremental monitoring is limited service level changes. That means, if one part of the source code of a service has changed, we instrument all the probes in the service. We proposed an approach (Section \ref{sec:extend the incremental SEFF reconstruction process}) that resolves this limitation. However, we could not evaluate it in the planed time of this thesis.    