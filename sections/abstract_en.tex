%% LaTeX2e class for student theses
%% sections/abstract_en.tex
%% 
%% Karlsruhe Institute of Technology
%% Institute for Program Structures and Data Organization
%% Chair for Software Design and Quality (SDQ)
%%
%% Dr.-Ing. Erik Burger
%% burger@kit.edu
%%
%% Version 1.3.2, 2017-08-01

\Abstract
Model-driven performance prediction can be used in order to predict software performance at the design stage. Applying it in the agile development process can reduce time and effort spent on monitoring. However, building such a model manually is impractical in the agile development process because it tends to generate quick releases in shot cycles. Existing approaches that extract automatically Performance Model (PM) are based on reverse-engineering and/or measurement techniques. However, these approaches require to monitor and analyze the whole system after each iteration, which will cause a monitoring overhead. 

The approaches that extract incrementally the PM from the source code and keep them consistent need a monitoring approach that provide them with the required monitoring information. Existing approaches either cannot provide fine-grained monitoring information or the instrumentation of the source code has to be done manually which is unsuitable for developers.

To overcome these problems, this thesis introduces an approach that can monitor the system in an adaptive way and provide PMs with the fine-grained monitoring information. Moreover, our approach is part of the continuous integration of PMs in the agile development process. We achieved this by introducing an automatic adaptive instrumentation approach of the source code. Furthermore, our instrumentation approach can monitor changes in the source code during system development and instrument the changed parts of the source code accordingly and automatically. During the system development, we collect automatically the monitoring probes that are needed to instrument the source code. When changes in the source code are committed, we automatically instrument the source code based on the monitoring probes that we've generated during the system development. The benefit of our approach will be to reduce the high monitoring overhead and to support the continuous integration of performance models.  

In particular, our approach can be integrated with other approaches that keep the source code and the PM consistent. While these approaches tend to keep the source code and its behavior in terms of PM consistent, our approach will provide them with the needed monitoring data to estimate the PM parameters (like Resource Demands RDs, loop execution number, the probability of selecting a branch).  

We evaluated our approach based on a case study. we showed that we could instrument incrementally and automatically the source code. Moreover, our evaluation showed that we could provide monitoring information that can be used for the calibration of performance model parameters considering parametric dependencies. Furthermore, using our approach and the case study we showed that we could reduce the monitoring overhead to 50\%. 