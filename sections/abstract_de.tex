%% LaTeX2e class for student theses
%% sections/abstract_de.tex
%% 
%% Karlsruhe Institute of Technology
%% Institute for Program Structures and Data Organization
%% Chair for Software Design and Quality (SDQ)
%%
%% Dr.-Ing. Erik Burger
%% burger@kit.edu
%%
%% Version 1.3.2, 2017-08-01
\Abstract
 
Die modellbasierte Performance-Vorhersage können zur frühzeitigen Abschätzung der	
Softwareperformanz verwendet werden. Der Einsatz solcher Werkzeuge	im	agilen	
Entwicklungsprozess	kann	den	in	die	Überwachung	investierten	Aufwand	reduzieren.	
Allerdings	ist	das	bauen	eines	solchen	Modells	während	des	agilen	Entwicklungsprozesses	
unpraktisch,	da	es	oft	dazu	führt,	Releases	in	kurzen	Intervallen	zu	produzieren.	Bisherige	
Ansätze,	welche	automatisch	ein	Performance-Modell	(PM)	generieren,	arbeiten	auf	
Grundlage	von	Reverse	Engineering	und/oder	Messverfahren.	Allerdings	überwachen	und	
analysieren	diese	Ansätze	das	gesamte	System	nach	jeder	Iteration,	was	einen	
Überwachungsaufwand	erzeugt.

Die	Ansätze,	welche	inkrementell	das	PM	aus	dem	Quellcode	extrahieren	und	sie	konsistent	
halten,	sind	auf	einen	Überwachungsansatz	angewiesen,	welcher	sie	mit	den	benötigten	
Informationen	versorgt.	Vorhandene	Ansätze	können	entweder	bei	der	Überwachung	keine	
feingranulare	Informationen	liefern	oder	die	Instrumentierung	des	Quellcodes	muss	
manuelle	vorgenommen	werden,	was	den	Entwicklern	nicht	zumutbar	ist.

Um	diese	Probleme	zu	überwinden,	stellt	diese	Thesis	einen	Ansatz	zur	adaptiven	
Systemüberwachung	vor,	welches	PMen	finegranulare	Überwachungsinformationen	zur	
Verfügung	stellt.	Zudem	ist	unser	Ansatz	Teil	der	kontinuierlichen	Integration	von	PMen	in	
den	agilen	Entwicklungsprozess.	Das	haben	wir	durch	die	Einführung	eines	automatischen	
adaptiven	Instrumentierungsprozesses	des	Quellcodes	erreicht.	Außerdem	kann	unser	
Instrumentierungsansatz	Quellcodeänderungen	während	der	Systementwicklung	
überwachen	und	die	veränderten	Teile	des	Quellcodes	entsprechend	automatisch	
instrumentieren.	Während	der	Systementwicklung	sammeln	wir	automatisch	die	
Überwachungsproben,	welche	zur	Instrumentierung	des	Quellcodes	benötigt	werden.	Wenn	
Änderungen	im	Quellcode	vorgenommen	werden,	instrumentieren	wir	den	Quellcode	
automatisch	basierend	auf	den	Überwachungsproben,	welche	wir	während	der	
Systementwicklung	generiert	haben.	Der	Vorteil	unseres	Ansatzes	wird	die Reduzierung	des	
hohen	Überwachungsaufwandes	und	die	Unterstützung	der	kontinuierlichen	Integration	der	
PMe	sein.

Insbesondere	kann	unser	Ansatz	mit	anderen	Ansätzen,	welche	den	Quellcode	und	das	PM	
konsistent	halten,	zusammengeschlossen	werden.	Während	jene	Ansätze	tendenziell	den	
Quellcode	und	ihr	Verhalten	bezüglich	des	PMs	konsistent	halten,	wird	unser	Ansatz	die	
notwendigen	Überwachungsdaten	liefern,	um	die	PM	Parameter	zu	schätzen	(wie	
Ressourcenbeanspruchung,	Anzahl	von	Schleifenausführungen,	Wahrscheinlichkeit	einer	
Programmierverzweigung).

Wir	haben	unseren	Ansatz	anhand	einer	Fallstudie	evaluiert.	Wir	haben	gezeigt,	dass	wir	den	
Quellcode	inkrementell	und	automatisch	instrumentieren	konnten.	Außerdem	hat	unsere	
Evaluierung	gezeigt,	dass	wir	Überwachungsinformation	zur	Verfügung	stellen	konnten,	
welche	für	die	Kalibrierung	der	PM	Parameter	benutzt	werden	können,	unter	
Berücksichtigung	von	parametrischen	Abhängigkeiten.	Weiterhin	haben	wir	mit	unserem	
Ansatz	und	der	Fallstudie	gezeigt,	dass	wir	den	Überwachungsaufwand	auf	50\%	reduzieren	
konnten.